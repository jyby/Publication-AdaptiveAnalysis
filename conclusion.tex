%%
%% conclusion.tex
%% 
%% Made by Jeremy Barbay
%% Login   <jbarbay@condorito>
%% 
%% Started on  Thu Apr 10 17:40:00 2008 Jeremy Barbay
%% Last update Thu Apr 10 17:40:00 2008 Jeremy Barbay
%%

% Conclusion and Perspectives

\chapter{Conclusion and Perspectives}
\label{cha:conclusion}

% What we covered

In this course I tried to cover the various applications of the idea
``size is not the only difficulty measure possible'', to various
problems.
%
This is a huge work and I was barely able to skim the surface of it,
but my experience showed me that the interaction between the different
fields which used this idea is fruitful.

\tinyskip 

% What can be done in the future. / Why I didn't do this.


Among the fruits I hope to ripe from this work:
\begin{itemize}
\item Adaptive Analysis techniques applied to Computational Geometry,
  in order to yield even better analysis and algorithms than mere
  output sensitive ones;

\item Apply the formalism from Parameterized Complexity to all
  problems, so that one can easily define reductions (even polynomial)
  between pairs of problems and difficulty measures.

\item Apply adaptive analysis to the space complexity, and in
  particular to compression (the entropy is just a difficulty measure
  suggested by information theory) and succinct indexes. In
  particular, I hope to exploit the duality between search and coding
  algorithms, obvious between integer encodings and searching in
  sorted array, but less obvious on more complex structures.
\end{itemize}

I will co-organize a workshop in April 2009 at Dagstuhl, Germany (19
to 24), on ``Adaptive, Output Sensitive, Online and Parameterized
Algorithms'', with Rolf Klein (Universit{\"a}t Bonn, DE), Alejandro
Lopez-Ortiz (University of Waterloo, CA) and Rolf Niedermeier
(Universit{\"a}t Jena, DE). 
%
Some interesting interactions should occur then.



%%% Local Variables: 
%%% mode: latex
%%% TeX-master: "adaptiveAnalysisOfAlgorithm"
%%% End: 
