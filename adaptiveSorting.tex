%%
%% adaptiveSorting.tex
%% 
%% Made by Jeremy Barbay
%% Login   <jbarbay@condorito>
%% 
%% Started on  Fri Mar 21 15:54:07 2008 Jeremy Barbay
%% Last update Fri Mar 21 15:54:07 2008 Jeremy Barbay
%%

\chapter{Sorting}
\label{cha:adapt-analys-sort}



\section{In the Worst Case}
\label{sec:worst-case}

\subsection{Classical Algorithms}
\label{sec:classical-algorithms}


\begin{interaction}
  What sorting algorithm do you know? What is their ``complexity''?
  What type of complexity are we talking about?
\end{interaction}

There are many sorting algorithms in the comparison model, which can
be classified in two broad classes by their worst case complexity over
instances of size $n$: those in $\bigo(n\lg n)$, and those in
$\bigo(n^2)$.

\begin{itemize}
\item insertion sort $\bigo(n^2)$
\item quicksort $\bigo(n^2)$
\item bubblesort $\bigo(n^2)$
\item heapsort $\bigo(n\lg n)$
\item selection sort $\bigo(n\lg n)$
\item (...)
\end{itemize}

\begin{interaction} 
  Why can't we improve the worst case complexity (in the
  comparison model)? How do you prove a lower bound?
\end{interaction}


The complexity of $\bigo(n\lg n)$ is optimal in the comparison model,
as it is easy to prove a lower bound of $\Omega(n\lg n)$ on the depth
of any decision tree sorting arrays of size $n$.


\begin{interaction}
  Does it mean that all $\bigo(n\lg n)$ sorting algorithms are
  equivalent in practice? Then what makes them different?
\end{interaction}

But even on a topic as old and mature as sorting, there is a first gap
between theory and practice: while the algorithms whose worst case
complexity over instances of size $n$ is $\bigo(n\lg n)$ are
theoretically better, in practice they don't perform equivalently, and
for some applications algorithms with a larger worst case complexity
are performing better.


\begin{interaction}
  Which sorting algorithm you know is ``adaptive'', in the sense that
  it has a great variation of complexity, from ``easy'' instances to
  ``hard instances''?
\end{interaction}


Presortedness can be used in sorting, in particular fpr the following
measures of order on an input $X$ to be sorted:

\begin{itemize}

\item \emph{Inversions}, the number of pairs in the wrong order,
  defined more formally as
$$\Inv(X)=|{(i,j) | 1\leq i<j\leq n, x_i>x_j}|,$$
was used as a measure of presortedness by Mehlhorn~\cite{mehlhorn} and
Guibas~\etal~\cite{guibas}.

\item \emph{Runs}, the number of ascending substrings of $X$
  defined more formally as
  $$\Runs(X)=|{i | 1\leq i\leq n, x_{i+1}>x_i}|,$$
  was used as a measure of presortedness by
  Knuth~\cite{theArtOfComputerProgrammingVol3}.

\item \emph{REM}, the number of elements which have to be romved from
  $X$ to leave a sorted list, the complement of the length of the
  longest ascending subsequence of $X$,
  defined more formally as
  $$\Rem(X)= n-\max \{t | \exists 1\leq i_1<\ldots<i_t\leq n | x_{i_1} < x_{i_t} \},$$
  was used as a measure of presortedness in an empirical study by Cook
  and Kim~\cite{cook}, and more theoretically by
  Mannila~\cite{measuresOfPresortednessAndOptimalSortingAlgorithms}.

\item \emph{EXC}, the smallest number of exchanges of arbitrary
  elements needed to bring $X$ into ascending order, ,
  defined more formally as
  $$\Exc(X)= n-\mbox{number of cycles in the permutation of $\{1,\ldots,n\}$ corresponding to $X$.},$$

\end{itemize}



\subsection{Insertion Sorts}
\label{sec:insertion-sorts}



\subsubsection{Straight Insertion Sort}
\label{sec:stra-insert-sort}

The straight insertion sort consists to update the left part of the
subarray by inserting in it each new element in linear time. 
%
The overall complexity is $\bigo(n^2)$ comparisons, but it is analyzed
in a finer way by counting the number of inversions $\idtt{Inv}(X)$,
defined as follow:

$$\idtt{Inv}(X)=\#\{(i,j), i<j, X_i > X_j\}$$

It is straightforward to see that the complexity of straight insertion
sort is exactly $\idtt{Inv}(X)$, which is $\bigo(n^2)$.

\subsubsection{Local Insertion Sort}
\label{sec:local-insertion-sort}

See Estivil-Castro and Wood's survey~\cite[page
5,6]{estivillcastro92survey} or the original paper by
Mannila~\cite{measuresOfPresortednessAndOptimalSortingAlgorithms} for
the analysis of Local Insertion sort using a Dynamic Finger Search
Tree, called a ``level-linked balanced
$(a,b)$-tree''~\cite{designAndAnalysisOfADataStructureForRepresentingSortedLists},
or any similar technique allowing to maintain a dictionary while
supporting local
searches~\cite{aNewRepresentationForLinearLists,fingerSearchTreesWithConstantInsertionTime,distancesAndFingerSearchInRandomBinarySearchTrees,spaceEfficientFingerSearchOnDegreeBalancedSearchTrees}.

\begin{adaptiveanalysis}[e]
  \begin{tabular}{@{\bf}p{.25\textwidth}p{.75\textwidth}}
    Problem            & {\tt Sorting Permutations}\\               
    Classical Approach & With {\tt Heapsort}, with $\bigo(n\log n)$ comparisons and space $\bigo(1)$. \\
    Easy Instances     & For {\tt Straight Insertion sort}, from $n$ to $n(n-1)/2$ comparisons.\\
    Difficulty Measure & $\idtt{Inv}(X)=\#\{(i,j), i<j, X_i > X_j\}$\\
    Adaptive Analysis  & {\tt Straight Insertion sort} performs $\bigo( n + \idtt{Inv}(X) ) $ \\
    New Algorithm      & {\tt Local Insertion sort} performs $\bigo( n ( 1 + \lg( \idtt{Inv}(X) / n ) ) )$ comparisons 
    and uses $\bigo(n)$ space with a dynamic finger search tree. \\
    Lower Bound        & $\Omega( n ( 1 + \lg( \idtt{Inv}(X) / n ) ) )$ comparisons are required. \\
  \end{tabular}
  \caption{Worst Case Adaptive Analysis of Permutation Sorting with
    Comparisons}
  \label{tab:permutationSortingWorstCase}
\end{adaptiveanalysis}


\subsection{Partition Sorts}
\label{sec:partition-sorts}

\subsubsection{Merge Sort}
\label{sec:merge-sort}

Another traditional algorithm is merge sort.
%
This algorithm is based on a simple linear procedure to merge two
already sorted arrays.
%
As each subarray of size one is already sorted, it is clear that in
$\lg n$ merging phases taking linear time each, the whole array can be
sorted in time $\bigo(n\lg n)$.

\subsubsection{Runs Sort}
\label{sec:runs-sort}

Yet, if some subarrays of larger size are known to be sorted, the
merge sort algorithm can go faster.
%
For instance, checking in linear time for \emph{down-step} positions
in the array, where an element is followed by a smaller one,
partitions the original arrays in ascending runs which are already
sorted.

\begin{definition}
  A \emph{down step} of a permutation $\pi$ over $[n]$ is a position $i$ 
  such that $\pi(i+1)<\pi(i)$.  
  % 
  A \emph{run} in a permutation $\pi$ is a maximal range of
  consecutive positions $\{i,\ldots,j\}$ which does not contain any
  down step.
  % 
  Let $d_1,d_2, \ldots,d_k$ be the list of consecutive down steps in $\pi$.
  Then the number of maximal runs of $\pi$ is noted $\nRuns = k+1$, 
  and the sequence of the lengths of the maximal runs is noted
  $\Runs = \langle d_1,d_2-d_1, \ldots,d_k-d_{k-1},n+1-d_k \rangle$.
\end{definition}

\begin{definition}
  The {\em entropy} of a sequence of integers $X=\langle n_1,n_2,
  \ldots, n_\nRuns\rangle$ adding up to $n$ is $H(X) =
  \sum_{i=1}^\nRuns\frac{n_i}{n} \lg \frac{n}{n_i}$.
  % 
  By the convexity of logarithm we have $H(X) \le \lg \nRuns$ and
  $H(X) \ge \frac{(\nRuns-1)(1/\ln 2 + \lg n)}{n}$.
  % (r-1)lg n + (n-r+1)lg n/(n-r+1)
  % (n-r+1)lg (1+(r-1)/(n-r+1))
  % (r-1)/ln 2
\label{def:entrop}
\end{definition}

\begin{theorem}
  There is an algorithm performing at most $n(2+H(\Runs))(1+o(1)) +
  \bigo(\nRuns\log n)$ comparison to sort a permutation $\pi$ over
  $[n]$ with a set $\Runs$ of $\nRuns$ maximal run lengths.
\end{theorem}

Obviously, the same concept can be applied to other partitioning of
the original array, such as alternating sequences of ascending and
descending runs, or mixes of ascending and alternating runs.

\subsubsection{Shuffled Monotone Subsequences  sort}
\label{sec:sms-sort}

A more complicated parititioning scheme is based on \emph{Shuffled
  Monotone Subsequences} (SMS), considering subsequences of
non-necessarily consecutive positions in the array.

\begin{definition}
  A \emph{Shuffled Monotone Subsequence} (resp. a \emph{Strict
    Shuffled Monotone Subsequence}) of a permutation $\pi$ over $[n]$
  is a subsequence $\{i,\ldots,j\}$ of positions forming a run
  (resp. strict run).
  % 
  The number of maximal shuffled monotone subsequences of $\pi$ is
  noted $\nSMS$ (resp. $\nSSMS$), and the sequence of the lengths of
  the maximal shuffled monotone subsequences is noted $\SMS$
  (resp. $\SSMS$)
\end{definition}

Shuffled monotone subsequences capture more of the structure in the
permutation, at the cost of a more costly precomputing step:
\begin{lemma}
  Given a permutation $\pi$ over $[n]$ with $\nSMS$ maximal shuffled
  monotone subsequences, there is an algorithm finding one set $\SMS$
  (among the potentially many maximal ones) of $\nSMS$ maximal
  shuffled monotone subsequences in time $\bigo(n\lg(\nSMS+1))$.
\end{lemma}

\begin{theorem}
  There is an algorithm performing at most $n
  \lceil\lg\nSMS\rceil(1+o(1)) +O(\log n)$ comparison to sort a
  permutation $\pi$ over $[n]$ with a set of $\nSMS$ maximal shuffled
  monotone subsequences.  
\end{theorem}


\subsubsection{Removals}
\label{sec:removals}

Mannila~\cite{measuresOfPresortednessAndOptimalSortingAlgorithms}

\subsubsection{Other Measures of Difficulty}
\label{sec:other-meas-diff}

We described here only a representative selection of the three main
types of adaptive sorting algorithms.
%
See Estivil-Castro and Wood's survey~\cite{estivillcastro92survey} for
the description of the other difficulty measures and corresponding
adaptive algorithms.




\subsection{Adaptive Lower Bound}
\label{sec:adaptive-lower-bound}


\subsection{Reductions Between Analysises}
\label{sec:reduct-betw-analys}


Comparisons and reductions between difficulty measures, graph of
reductions.

See Manilla's paper.








\section{In the Average Case}
\label{sec:other-types-adaptive}

\subsection{Average Case and Adaptive Randomized Algorithm}
\label{sec:aver-case-adapt}

\subsection{Random Local Insertion}
\label{sec:rand-local-insert}



\begin{adaptiveanalysis}[e]
  \begin{tabular}{@{\bf}p{.25\textwidth}p{.75\textwidth}}
    Problem            & {\tt Randomized Sorting Permutations}\\               
    Classical Approach & With {\tt Heapsort}, with $\bigo(n\log n)$ comparisons and space $\bigo(1)$. \\
    Easy Instances     & .\\
    Difficulty Measure & $\idtt{Inv}(X)=\#\{(i,j), i<j, X_i > X_j\}$\\
    Adaptive Analysis  & {\tt Straight Insertion sort} performs $\bigo( n + \idtt{Inv}(X) ) $ \\
    New Algorithm      & {\tt Skip sort} performs $\bigo( n ( 1 + \lg( \idtt{Inv}(X) / n ) ) )$ comparisons 
    and uses $\bigo(n)$ space with a randomized skip list. \\
    Lower Bound        & $\Omega( n ( 1 + \lg( \idtt{Inv}(X) / n ) ) )$ comparisons are required. \\
  \end{tabular}
  \caption{On Average Adaptive Analysis of Permutation Sorting with
    Comparisons}
  \label{tab:permutationSortingRandomized}
\end{adaptiveanalysis}


\section{Perspectives}
\label{sec:perspectives}

The adaptive analysis of sorting in the comparison model is one of the
domain where the adaptive analysis are the most complete. It
illustrates the possibility of various difficulty measures, adaptive
lower bound, reductions between difficulty measures. It can be a model
for the adaptive analysis of other problems.

Estivil-Castro and Wood~\cite{estivillcastro92survey} also mention
adaptive analysis over {\em distributions} and in other models, such
as in external memory. 
%
Those fields are less mature, and a source of
open problems. 
%
All those results are still in the comparison model.

\begin{homework}[e]
\caption{Reading on Adaptive Sorting.}
  Read Estivil-Castro and Wood's survey~\cite{estivillcastro92survey}.
\end{homework}

\begin{homework}[e]
\caption{Adaptive Analysis of BucketSort. \label{hmw:bucketSort}}
  Consider bucket-sort, an algorithm which is not in the comparison
  algorithm. Give a difficulty measure and an adaptive analysis of its
  complexity.
  \begin{solution}
    Difficulty measures would be the (known in advance) size of the
    domain of values, and the maximum number of elements with the same
    value.
  \end{solution}
\end{homework}

\begin{openproblem}
\caption{Adaptive Analysis of value-based algorithms.\label{opb:valueBasedSorting}}
  Is there a difficulty measure and a corresponding adaptive analysis
  for sorting algorithms which are based on the values of the
  elements?
\end{openproblem}

Sorting Multisets is often easier than sorting partition, and
Kirkpatrick and Seidel~\cite{kirkpatrick} showed how sorting multisets
is equivalent to computing the convex hull of a set of planar points:
we describe the work on convex hulls in the next chapter.



% \section{Homeworks}
% \label{sec:homeworks}


% \inputProblem{./}{multisetInvSorting01.pbtex}{3 marks}
% \inputProblem{./}{countingSort01.pbtex}{6 marks}



%%% Local Variables: 
%%% mode: latex
%%% TeX-master: "adaptiveAnalysisOfAlgorithm"
%%% End: 
