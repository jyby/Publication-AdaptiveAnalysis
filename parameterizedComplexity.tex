%%
%% parameterizedComplexity.tex
%% 
%% Made by Jeremy Barbay
%% Login   <jbarbay@condorito>
%% 
%% Started on  Tue May  6 19:33:59 2008 Jeremy Barbay
%% Last update Tue May  6 19:33:59 2008 Jeremy Barbay
%%

\chapter{Parameterized Complexity}
\label{cha:param-compl}



\section{NP}
\label{sec:np}

Reminder from previous courses


NP-Hard problems known by students:
\begin{itemize}
\item SAT, 3SAT

\item Vertex Cover Problem

\item (...)

\end{itemize}



We will use the Vertex Cover problem to illustrate the concepts of
this chapter.

\begin{definition}
  A \emph{vertex cover} for an undirected graph $G = (V,E)$ is a
  subset $S$ of its vertices such that each edge has at least one
  endpoint in $S$. 
%
  In other words, for each edge $(a,b)$ in $E$, one of $a$ or $b$ must
  be an element of $S$.
\end{definition}

\begin{definition}
The \emph{vertex cover problem} is the optimization problem of finding a vertex cover of size k in a given graph.
\begin{itemize}
\item INSTANCE: Graph G.
\item OUTPUT: Smallest number k such that there is a vertex cover S for G of size k.
\end{itemize}

Equivalently, the problem can be stated as a decision problem:
\begin{itemize}
\item INSTANCE: Graph G and positive integer k.
\item QUESTION: Is there a vertex cover S for G of size at most k?
\end{itemize}
  
\end{definition}

\section{Multiple Input Parameters}
\label{sec:mult-input-param}

Even at the time of the definition of NP hard problems, it was known
that some very large instances of NP hard problems were still easy to
solve, and that those might be the one occuring in practice (that's
why Levin~\cite{levin} defined reductions on Average).



Parameterized Complexity characterizes the ``easier'' instances by
some parameters, which can help the algorithm to solve faster






\section{Perspective}
\label{sec:perspective-parameterized-complexity}

Alex Lopez-Ortiz says that some people working in the domain of
parameterized complexity have considered polynomial problems.
%
It is important to note that parameterized complexity, when applied to
problems which can be solved in polynomial worst case complexity in
the size of the input, is exactly adaptive analysis:
\begin{itemize}
\item The fact that in Parameterized Complexity the parameters are
  \emph{given} to the algorithm while traditionaly the difficulty of
  the instance is not given to an adaptive algorithm is \emph{not} a
  relevant difference of approach between parameterized complexity and
  adaptive algorithms: as long as the complexity of an algorithm is
  exponential in some parameter $k$, it does not matter if this
  parameter is given or not because it can be guessed at the mere cost
  of a constant factor in the complexity
  ($1+2+4+8+16+\ldots+2^k=2^{k+1}-1$).
%
  \begin{NOTE}
    This sounds obvious but I had to convince Jonathan Buss of it. As
    he is one of the figure of parameterized complexity, I guess it is
    worth mentionning...
  \end{NOTE}

\item Similarly, the fact that Parameterized Complexity uses reduction
  as opposed to lower bound is not a difference in definition of
  technique, but simply a difference in application: for problems in
  $P$ one can prove lower bound while for problems in $NP$ one proves
  reductions, parameterized or not.

\end{itemize}


On the other hand, I found the notation and formalism of reductions
defined in the field parameterized complexity very usefull to consider
reductions between polynomial problems, \emph{along with their
  difficulty measure}.




%%% Local Variables: 
%%% mode: latex
%%% TeX-master: "adaptiveAnalysisOfAlgorithm"
%%% End: 
