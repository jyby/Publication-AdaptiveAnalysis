%%
%% convexHull.tex
%% 
%% Made by Jeremy Barbay
%% Login   <jbarbay@condorito>
%% 
%% Started on  Thu Apr 10 17:40:00 2008 Jeremy Barbay
%% Last update Thu Apr 10 17:40:00 2008 Jeremy Barbay
%%

\chapter{Convex Hull}
\label{cha:convex-hull}

Given a finite set~$S$ of points, the {\em convex hull} is the
smallest convex polygon containing~$S$.
%
% From wikipedia http://en.wikipedia.org/wiki/Convex_hull
The problem of finding convex hulls finds its practical applications
in pattern recognition, image processing, statistics and GIS. It also
serves as a tool, a building block for a number of other
computational-geometric algorithms.
%
\begin{LONG}
  For example, consider the problem of finding the diameter of a set
  of points, which is the pair of points a maximum distance apart.
  %
  The diameter will always be the distance between two points on the
  convex hull.
  %
  The $\bigo(n\lg n)$ algorithm for computing the diameter proceeds
  by first constructing the convex hull, then for each hull vertex
  finding which other hull vertex is farthest away from it.
  %
  This so-called "rotating-calipers" method can be used to move
  efficiently from one hull vertex to another.


  \begin{table}
    \centering
    \begin{tabular}{ccp{5cm}cc}
      1972 &
      Graham's Scan &
      An efficient algorithm for determining the convex hull of a finite planar set. &
      $\bigo(n\lg n)$ &
      2D
      \\
      1973 &
      Jarvis's March &
      On the identification of the convex hull of a finite set of points in the plane.
      $\bigo (nh)$  &
      2D
      \\
      1977&
      Prepata and Hong &
      Convex Hulls of Finite sets of points in 2 or 3 dimensions. &
      $\bigo(n\lg n)$ &
      3D
      \\
      1986 &
      Kirkpatrick and Seidel \cite{kirkpatrick}&
      The ultimate planar convex hull algorithm? &
      $\bigo(n\lg h)$ &
      2D
      \\
      1989 &
      Clarkson and Shor &
      Applications of random sampling &
      $\bigo(n\lg h)$ on average &
      3D
      \\
      1991 &
      Edelsbrumer and Shi &
      An $\bigo(n\lg^2 h)$ algorithm for 3 dimension convex hull. &
      $\bigo(n\lg^2 h)$ &
      3D
      \\
      1995 &
      Chan, Snoeyink and Yap \cite{chan95outputsensitive} &
      Output sensitive construction of polytopes in four dimensions and clipped Vorono\"i diagrams in 3 dimensions. &
      $\bigo(n\lg h)$ &
      3D
      \\
      1995 &
      Chazelle and Matou\v{s}ek &
      Derandomizing an output sensitive convex hull algorithm in 3 dimensions. &
      $\bigo(n\lg h)$ &
      3D
      \\
      1996 &
      Chan \cite{chan96optimal}&
      Optimal output-sensitive convex hull algorithms in two and three dimensions &
      $\bigo(n\lg h)$ simpler &
      2D,3D
      \\
    \end{tabular}
    \caption{Bibliography of the Convex Hull}
    \label{tab:biblioConvexHull}
  \end{table}

There is an interesting tutorial on the convex hull  on the web: \url{http://www.cip.ifi.lmu.de/~viermetz/cg/}

\end{LONG}


Computing the convex hull is equivalent to computing the lower and
{\em upper hull}, as one can deduce both from the convex hull, and as
the convex hull is the union of both.
%
The easiest case is when the points are two
dimensions~\cite{kirkpatrick,jarvis73,graham72}, but efficient
output-sensitive algorithms are known for three
dimensions~\cite{optimalOutputSensitiveConvexHullAlgorithmsInTwoAndThreeDimensions}
and even four dimensions~\cite{DBLP:journals/dcg/ChanSY97}.



\section{Planar Convex Hull}
\label{sec:planar-convex-hull}


\subsection{Graham's scan:  $\bigo(n\lg n)$}
\label{sec:grahams-scan}


Graham~\cite{graham72}'s scan algorithm computes the planar convex
hull from a sorted set of vertices in linear time, which yields an
$\bigo(n\lg n)$ time general algorithm, hence improving previous
results in the worst case among instances of size $n$.
%
Its complexity is not sensitive to the size $h$ of the convex hull,
but it is sensible to the order of the input when using an adaptive
algorithm.

\begin{algorithm}
  \begin{algorithmic}
    \STATE Test
  \end{algorithmic}
\end{algorithm}

\subsection{Jarvis's Walk: $\bigo (nh)$}
\label{sec:jarviss-walk}


Jarvis~\cite{jarvis73} proposed the first non-trivial algorithm to
compute the convex hull in the plane.
%
It ``wraps around'' the set, finding the next edge of the convex hull
by comparing the slopes of the lines between the last vertex added to
the convex hull and the remaining vertices.
%
According to Jarvis's original analysis it performs $\bigo(n^2)$ such
operations in total to compute the convex hull of a set of $n$ planar
points, but in fact its time complexity is bounded by $\bigo (nh)$,
where $h$ is the size of the convex hull.



\subsection{Kirkpatrick and Seidel's algorithm: $\bigo(n\lg h)$}
\label{sec:kirkp-seid-algor}


This was improved by Kirkpatrick and Seidel~\cite{kirkpatrick}, who
proposed an algorithm computing the convex hull in $\bigo(n\lg h)$,
introduced the concept of {\em output sensitive} complexity, and
proved a tight lower bound on the worst case complexity among
instances of size $n$ and ouput-size $h$.
%
See the very good wikipedia entry at
\verb+http://en.wikipedia.org/wiki/Kirkpatrick%E2%80%93Seidel_algorithm+.


\begin{adaptiveanalysis}[e]
  \begin{tabular}{@{\bf}p{.25\textwidth}p{.75\textwidth}}
    Problem            & {\tt Planar Convex Hull}\\               
    Classical Approach & With Graham's scan, of complexity $\bigo(n\log n)$. \\
    Easy Instances     & Small convex Hull, linear with Jarvis's walk.\\
    Difficulty Measure & $h$, the size of the convex hull\\
    Adaptive Analysis  & Jarvis's walk  performs $\bigo(nh)$ comparisons. \\
    New Algorithm      & Kirkpatrick and Seidel's algorithm  performs $\bigo(n\log h)$. \\
    Lower Bound        & $\Omega( n \log h)$. \\
  \end{tabular}
  \caption{Worst Case Adaptive Analysis of Planar Convex Hull
    Computation}
  \label{tab:planarConvexHull}
\end{adaptiveanalysis}

\subsection{Chan's algorithm: $\bigo(n\lg h)$, but simpler, and in $3$ dimensions}
\label{sec:chans-algor-:big}


Chan~\cite{optimalOutputSensitiveConvexHullAlgorithmsInTwoAndThreeDimensions}
simplified the output-sensitive algorithm by Kirkpatrick and Seidel
and extended it to three dimensions through a grouping method, so that
it runs in optimal $\bigo(n\lg h)$ time.
%
\begin{LONG}
  It uses a grouping trick, Dobjin-Kirkpatrick's hierachies and a
  guessing technique from Chazelle and Matou\v{s}ek.
\end{LONG}
%
The paper is online on Chan's webpage, and you can see the good
wikipedia entry at
\verb+http://en.wikipedia.org/wiki/Chan%27s_algorithm+.


\begin{adaptiveanalysis}
  \begin{tabular}{@{\bf}p{.25\textwidth}p{.75\textwidth}}
    Problem            & {\tt 3D Convex Hull}\\               
    Classical Approach & With Prepata and Hong's algorithm, of complexity $\bigo(n\log n)$. \\
    Easy Instances     & Small convex Hull, linear with gift-wrapping algorithm.\\
    Difficulty Measure & $h$, the size of the convex hull\\
    Adaptive Analysis  & gift-wrapping algorithm  performs $\bigo(nh)$ comparisons. \\
    New Algorithm      & Chan's algorithm  performs $\bigo(n\log h)$. \\
    Lower Bound        & $\Omega( n \log h)$. \\
  \end{tabular}
  \caption{Worst Case Adaptive Analysis of 3D Convex Hull Computation}
  \label{tab:planarConvexHull}
\end{adaptiveanalysis}


\subsection{Extensions in other models}
\label{sec:extens-other-models}



Goodrich~\etal~\cite{externalMemoryComputationalGeometry} presented an
optimal output-sensitive two-dimensional convex hull algorithm, which
makes $O(N/B \log_{M/B} H/B)$ cache misses in the external model,
where $N$ is the number of points, $M$ is the cache size, $B$ is the
block size, and $H$ is the number of hull vertices.
%
In turn, Afshani and
Farzan~\cite{cacheObliviousOutputSensitiveTwoDimensionalConvexHull}
generalized the result to the cache-oblivious model.


\section{Other Output sensitive algorithms in Computational Geometry}
\label{sec:other-outp-sens}

\begin{verbatim}
Verbatim from Almr Elmasry

Nielsen \cite{n} formalized the idea of Chan's convex hull algorithm
in a paradigm that he called grouping-and-querying.  
%
The idea is to break these algorithms into two stages: estimating the
output size, and then building data structures based on that estimate
which are queried to construct the final solution.  
%
Nielsen used this paradigm for computing convex hulls of objects in
the plane \cite{n2}, maximal and convex layer decomposition \cite{n3},
spatial point-set matching \cite{n1}. 
%
Another application of the paradigm is used to produce specific cells
(by identifying the corresponding cites) of the two-dimensional
Voronoi diagram in $O(n \log{v})$ \cite{n}, where $v$ is the output
size.


Another classical example is the problem of reporting the
intersections of a set of $n$ line segments in the plane. 
%
Balaban gave an optimal output-sensitive algorithm that runs in $O(n\log n+s)$, where $s$ is the number of such
intersections. 
%
Output-sensitive algorithms for several variants of the line-segment
intersection problem were also introduced \cite{a,bgr,ch,em}.

Bremner et al. \cite{bd} gave optimal output-sensitive algorithms for
computing the decision boundary of a bichromatic set of points in one
and two dimensions, by partitioning the space into regions closer to a
red point and those closer to a blue point. 
%
Their algorithms run in $O(n \log{b})$, where $b$ is the number of
points that contribute to the decision boundary.
\end{verbatim}

\begin{DISTRIBUTIONSENSITIVE}
  Several authors considered the point location problem under various
  assumptions about the query distribution.
%
  All solutions compare the expected query cost to the entropy bound
  $H=\sum p_i \log{(1/p_i)}$, where $p_i$ is the probability that the
  query point is contained in face $i$.
%
  When the distribution of the queries is given in advance, Arya et
  al. \cite{ar} have given a linear-space data structure for point
  location in triangulations.
%
  Their algorithm involves $H+O(H^{1/2}+1)$ point-line expected number
  of comparisons.
%
  Recently, a data structure is presented for point location in convex
  planar subdivisions when the distribution of the queries is given in
  advance \cite{co}.
%
  Their structure has an expected query time that differs from the
  optimal only for lower order terms in the linear-comparison tree
  model.
%
  On the other hand, for spatial distributions of point location
  queries, Iacono and Langerman \cite{il2} introduced a data structure
  that executes a point location query quickly if it is spatially
  close to the previous query.

  Sen and Gupta \cite{sg} gave distribution-sensitive algorithms for
  several geometric problems as the two-dimensional convex hull and
  maximal vector problems.
\end{DISTRIBUTIONSENSITIVE}



\section{Perspective}
\label{sec:perspective}

Although a lot of efforts have been put in optimizing the algorithms
for the convex hull and other problems in computational geometry
beyond the traditional worst case complexity, the only measure of
difficulty considered so far was the output size.
%
The complexity of the Graham's scan algorithm is trivially adaptive to
the order of the input when using an adaptive sorting algorithm, but
it seems superfluous that it depends as much on the relative order of
elements \emph{not} in the convex hull than to their relative order
with points of the convex hull: a better analysis should yield finer
measures of difficulty.

An interesting feature of the convex-hull is its similarity with the
problem of sorting \emph{multisets} (as opposed to its reduction to
the problem of sorting permutations). 
%
Adaptive analysises of the problem of sorting \emph{multisets} with
complete order should yield adaptive analysises of the convex hull
computation in the plane; and adaptive analysises of the problem of
sorting \emph{multisets} with a partial order could yield adaptive
analysises of the convex hull computation in the higher dimensions.

\begin{homework}
  \caption{Adaptive Analysis of Convex Hull}    
  Study the adaptive analysis of the convex hull of polygonal chains
  by Levcopoulos, Lingas and
  Mitchell~/cite{adaptiveAlgorithmsForConstructingConvexHullsAndTriangulationsOfPolygonalChains}.
\end{homework}


\begin{openproblem}
  \caption{Generalizing Sorting to Convex Hull}
  Choose a measure of difficulty for sorting, and generalize it to the
  convex hull. For instance, what would be a generalisation of the
  measure \texttt{Inv}, taking into account that the inversions
  between the points not in the convex hull does not matter, but that
  the inversions between the points in the convex hull and between the
  points in the convex hulls and the others do matter.
\end{openproblem}

\begin{enumerate}
\item Input order matters:
  \begin{itemize}
  \item whichever the number of dimensions, the points are given in
    sequential order;

  \item Chan already remarked than discarding points improve the
    algorithm in practice;

  \item With the right order, it is trivial to obtain a linear case
    complexity in the plane, and $dn$ in any number of dimension $d$.

  \end{itemize}

\item In 2D there is a link to sorting multisets with complete order.
In 3D or higher dimensions, there might be a link to sorting with partial order.


\item There is a large bibliography on sorting algorithms (for
  partitions with total order) taking advantage of input order. Not
  all results but at least some do generalize to multisets, and may
  generalize to partial order.

\end{enumerate}


%%% Local Variables: 
%%% mode: latex
%%% TeX-master: "adaptiveAnalysisOfAlgorithm"
%%% End: 
